\documentclass[a4paper, 12pt]{article}
\linespread{1.5}
\usepackage{mathtext}                   % русские буквы в фомулах
\usepackage[utf8]{inputenc}             % выбор кодировки
\usepackage[T2A,T1]{fontenc}
\usepackage[english, russian]{babel}    % русский язык
\usepackage[a4paper]{geometry}
\geometry{verbose,tmargin=1cm,bmargin=1.5cm,lmargin=1cm,rmargin=1cm}  %установка размеров
\usepackage{indentfirst}    % красная строка
\usepackage{amsmath}        % математические формулы
% \usepackage[LGRgreek]{mathastext}
\usepackage{amssymb}        % математические символы
\usepackage{multirow}       % объединение строк в таблице
\usepackage{graphicx}
\graphicspath{
  {pics/}
}
\usepackage[font=small,skip=5pt]{caption}
\renewcommand{\figurename}{Рис.}
\usepackage{float}
\usepackage{textcomp}       % типографские знаки
\usepackage{wrapfig}        % обтекаемые рисунки
\usepackage{mathrsfs}       % символы
\usepackage{fancyhdr}       % колонтитулы
\usepackage{colortbl}       % цветные таблицы
\usepackage{float}
\usepackage{cancel}         % перечеркнутый текст
%% Links
\usepackage{url}
\usepackage[unicode]{hyperref}

\usepackage{titlesec}
\titleformat{\section}{\normalsize\bfseries}{\thesection}{1em}{}
\titleformat{\subsection}{\normalsize\bfseries}{\thesubsection}{1em}{}

\begin{document}

% == LISTS FUNCTIONS ==
\newcommand{\listOn}{\begin{itemize}\setlength\itemsep{0pt}}
\newcommand{\listOff}{\end{itemize}}

\newcommand{\listNumOn}{\begin{enumerate}\setlength\itemsep{0pt}}
\newcommand{\listNumOff}{\end{enumerate}}

% == FIGURE FUNCTION ==
\newcommand{\pic}[5]
{
\begin{figure}[#1]
\begin{center}
\begin{minipage}{0.95\linewidth}
    \includegraphics[width=#2\linewidth]{#3}
    \caption{#4}
    \label{#5}
\end{minipage}
\end{center}
\end{figure}
}

% == TITLE ==
\begin{center}
    \large
    \bfseries
    Отчет по проекту ``Обратное проектирование антенн для терагерцовой оптоэлектроники'' за период 1 июня -- 14 августа \\
    \normalfont
    Михаил Лукьянов \\
\end{center}



\listOn
\item $\Gamma$ -- reflection coefficient (коэф отражения)
\item $Z_L$ -- load impedance (импеданс антенны)
\item $Z_0$ -- feeder charcteristic impedance (целевой импеданс)
\item $D(\theta, \phi, f)$ -- directivity (зависимость коэф отражения от углов ($\theta,\ \phi$) и частоты $f$)
\item $\lambda$ -- длина волны излучения
\item $\frac{\lambda^2}{4 \pi}$ -- эффективная площадь антенны
\listOff

\end{document}
